\documentclass{beamer}

\usepackage[spanish]{babel}
\usepackage{graphicx}
\usepackage{hyperref}
\usepackage[utf8]{inputenc}

\title{No existe el primo mas grande}
\author{Euclid of Alexandria}
\date{\today}

\begin{document}

\begin{frame}
	\titlepage
\end{frame}

\begin{frame}
	\frametitle{Resumen}
	\tableofcontents
\end{frame}

\section{Motivación}
\subsection{Punto de partida}
\begin{frame} 
	\frametitle{¿Qué son los números primos?}
	\begin{definition}
		En matemáticas, un \alert{número primo} es un número natural mayor que 1 que tiene
		únicamente dos divisores positivos distintos: él mismo y el 1.
	\end{definition}
	\begin{example}
		\begin{itemize}
			\item El 2 es un número primo \pause
			\item El 3 es un número primo \pause
			\item El 4 no es un número primo 
		\end{itemize}
	\end{example}
\end{frame}

\section{Resultados}
\begin{frame} \frametitle{Hay infinitos primos}
	\begin{block}{Prueba}
		Hola esto es una 
	\end{block} 
\end{frame}


\end{document}
