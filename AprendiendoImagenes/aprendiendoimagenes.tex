\documentclass[a4paper]{article}
\usepackage{ulem}
\usepackage[spanish]{babel}
\usepackage{graphicx}
\usepackage{wrapfig}
\usepackage[utf8]{inputenc}

\title{Aprendiendo imágenes}
\author{Carlos Aguilera Ventura}
\date{\today}

\begin{document}
\maketitle \tableofcontents \listoffigures \newpage

\section{Primeros pasos}
\includegraphics[width=\textwidth]{logolatex.png}


\section{Entornos flotantes}
hola que tal
\begin{figure}[!ht]
\centering
\includegraphics[width=100pt]{word.png}
\caption{Hola mundo}
\end{figure}

\section{Img y texto}
\begin{wrapfigure}[5]{r}{0.4\textwidth}
\includegraphics[width=100pt]{word.png}
\caption{Hello world}
\end{wrapfigure}


empezaba ya a cansarse de estar sentada con su
hermana a la orilla del río, sin tener nada que hacer:
había echado un par de ojeadas al libro que su hermana
Así pues, estaba pensando ({\tiny y pensar le costaba cierto
    esfuerzo, porque el calor del día la había dejado
soñolienta y atontada}) si el placer de tejer una
guirnalda de margaritas la compensaría del trabajo de
levantarse y coger las margaritas, cuando de pronto
saltó cerca de ella un Conejo Blanco de ojos rosados.






\end{document}
